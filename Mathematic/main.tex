%! TEX program = pdflatex




\documentclass[12pt]{article}

\usepackage[UTF8]{ctex}

\usepackage{answers}
\usepackage{setspace}
\usepackage{graphicx}
\usepackage{enumitem}
\usepackage{multicol}
\usepackage{mathrsfs}
\usepackage[margin=1in]{geometry} 
\usepackage{amsmath,amsthm,amssymb}
\usepackage{bm}     % 数学公式加粗
 
\newcommand{\N}{\mathbb{N}}
\newcommand{\Z}{\mathbb{Z}}
\newcommand{\C}{\mathbb{C}}
\newcommand{\R}{\mathbb{R}}

\DeclareMathOperator{\sech}{sech}
\DeclareMathOperator{\csch}{csch}
 
\newenvironment{theorem}[2][Theorem]{\begin{trivlist}
\item[\hskip \labelsep {\bfseries #1}\hskip \labelsep {\bfseries #2.}]}{\end{trivlist}}
\newenvironment{definition}[2][Definition]{\begin{trivlist}
\item[\hskip \labelsep {\bfseries #1}\hskip \labelsep {\bfseries #2.}]}{\end{trivlist}}
\newenvironment{proposition}[2][Proposition]{\begin{trivlist}
\item[\hskip \labelsep {\bfseries #1}\hskip \labelsep {\bfseries #2.}]}{\end{trivlist}}
\newenvironment{lemma}[2][Lemma]{\begin{trivlist}
\item[\hskip \labelsep {\bfseries #1}\hskip \labelsep {\bfseries #2.}]}{\end{trivlist}}
\newenvironment{exercise}[2][Exercise]{\begin{trivlist}
\item[\hskip \labelsep {\bfseries #1}\hskip \labelsep {\bfseries #2.}]}{\end{trivlist}}
\newenvironment{solution}[2][Solution]{\begin{trivlist}
\item[\hskip \labelsep {\bfseries #1}]}{\end{trivlist}}
\newenvironment{problem}[2][Problem]{\begin{trivlist}
\item[\hskip \labelsep {\bfseries #1}\hskip \labelsep {\bfseries #2.}]}{\end{trivlist}}
\newenvironment{question}[2][Question]{\begin{trivlist}
\item[\hskip \labelsep {\bfseries #1}\hskip \labelsep {\bfseries #2.}]}{\end{trivlist}}
\newenvironment{corollary}[2][Corollary]{\begin{trivlist}
\item[\hskip \labelsep {\bfseries #1}\hskip \labelsep {\bfseries #2.}]}{\end{trivlist}}

\usepackage[colorlinks=true, linkcolor=blue]{hyperref}

\usepackage{titlesec}
\usepackage{titletoc}
 
\begin{document}
 
% --------------------------------------------------------------
%                         Start here
% --------------------------------------------------------------
 
\title{A Daily Problem}%replace with the appropriate homework number
\author{ly\\ %replace with your name
Mathematic} %if necessary, replace with your course title
 
\maketitle
%Below is an example of the problem environment

\tableofcontents
%\titlecontents{section}  
%              [3cm]  
%              {\bf \large}%  
%              {\contentslabel{2.5em}}%  
%              {}%  
%              {\titlerule*[0.5pc]{$\cdot$}\contentspage\hspace*{3cm}}%  

\titlecontents{section}  
              [3cm]  
              {\bf \large}%  
              {\contentslabel{2.5em}}%  
              {}%  
              {\titlerule*[0.5pc]{$\cdot$}\contentspage\hspace*{3cm}}%  
\titlecontents{subsection}  
              [4cm]  
              {\bf \normalsize}%  
              {\contentslabel{2.5em}}%  
              {}%  
              {\titlerule*[0.5pc]{$\cdot$}\contentspage\hspace*{3cm}}%  



\clearpage


\section{函数极限与连续}

\begin{problem}{1}
    Let $\displaystyle f(x) = \prod_{k=2}^{n}\sqrt[k]{cosx}$, if $\displaystyle \lim_{x \to 0}\frac{1-f(x)}{x^2} = 10$, Find n. 
\end{problem}


\begin{solution}{1} \textbf{1}

    We use the L'Hospital rule, we get:

    \[
        \lim_{x\to 0} \frac{1-f(x)}{x^2} = \lim_{x \to 0} \frac{(tanx+tan2x+\cdots+tannx)f(x)}{2x} = \frac{1}{2}\sum_{k=2}^{n}k
        = \frac{(n-1)(n+2)}{4} = 10     
    \]
    thus $n = 6$.

    Note: We can use that : if $f(x) = f_1(x)f_2(x)\cdots f_n(x)$, so $\displaystyle f'(x) = \sum_{k=1}^{n} \frac{f_{k}^{'}(x)}{f_k(x)} f(x) $.
    
\end{solution}

\vspace{3cm}

\begin{problem}{2}
    Evaluate the limit:
    \[
        \lim_{n \to \infty} \frac{\sum_{k=1}^{n}\frac{6^k k!}{(2k+1)^k}}{\sum_{k=1}^{n} \frac{3^k k!}{k^k}}.
    \]
\end{problem}

\begin{solution}{2} \textbf{2}

    First use the ratio test to show that $\sum_{k=1}^{\infty}\frac{3^kk!}{k^k}$ diverges. Now you can apply the Stolz–Cesàro lemma. 
    The limit should be $\frac{1}{\sqrt{e}}.$
    
\end{solution}


\section{一元函数微积分}

\subsection{一元函数微分学}

\begin{problem}{1}
设 $f(x)$ 在 $(1,+\infty)$ 上连续可微, 且存在 $L>0$ 使得对 $\forall x, y \in (0, +\infty)$ 都有
\[
\mid f'(x)-f'(y) \mid < L\mid x-y\mid
\]
证明: $(f'(x))^2 < 2Lf(x)$.

\end{problem}

\begin{solution}

Waiting...
\end{solution}

\vspace{3cm}

\begin{problem}{2}
    已知$\sinh x = x\cosh y$, $x, y \in \left(0, 1\right)$, 证明$y<x<2y$.
\end{problem}


\begin{solution}{2} \textbf{2}

    由拉格朗日中值定理:
    \[
        \sinh x - \sinh 0 = x\cosh \xi , \xi\in \left(0, x\right)    
    \]    
    又$\sinh x = x\cosh y $, 有:
    \[
        x\cosh y = x\cosh \xi , x\in\left(0, 1\right)
    \]
    \[
        \cosh y = \cosh \xi , x\in\left(0, 1\right)
    \]
    又$\cosh x$在$\left(0, 1\right)$上递增, 所以有$y = \xi < x$.

    由$\cosh x = \frac{e^x + e^{-x}}{2} > \frac{2\sqrt{e^x\cdot e^{-x}}}{2} = 1$, 所以有:
    \[
        \int_0^x \cosh x \mathrm{d}x > \int_0^x 1\mathrm{d}x = x
    \]
    用$\frac{x}{2}$代替$x$可得到$\sinh \frac{x}{2} > \frac{x}{2}$, 两边同时乘上一个$\cosh x$, 得:
    \[
        \cosh x \sinh x > \frac{x}{2}\cosh x    
    \]
    又$\sinh x = 2\sinh \frac{x}{2}\cosh\frac{x}{2}$, 有:
    \[
        \sinh x > x\cosh \frac{x}{2}     
    \]
    又$\sinh x = x\cosh y$, 有
    \[
         x\cosh y > x\cosh \frac{x}{2}, x\in\left(0, 1\right)  
    \]
    即
    \[
        \cosh y > \cosh \frac{x}{2}
    \]
    又$\cosh x$在$\left(0, 1\right)$上严格单增, 所以有$y>2x$.\\
    综上, $y<x<2y$.
\end{solution}


\pagebreak









\subsection{一元函数积分学}

\begin{problem}{1}

    Suppose $ f: [0, 1] \to \mathbb{R} $ is continuous in $[0,1]$ and differentiable in $(0,1)$. Supoppse $f(0) = 0$ and 
    $0<f'(x)\leq 1$ for all $x \in (0,1)$. 
    \item (a)Prove that the function $\Phi (x) = (\int_{0}^{x} f(t)dt)^2 - \int_{0}^{x} f^3(t)dt$ is monotone.
    \item (b)Find all functions $f$ such that
        \[
            (\int_{0}^{1} f(t) dt)^2 = \int_{0}^{1} f^3(t) dt.
        \]
    
\end{problem}


\begin{solution}{1} \textbf{1}

    \item (a) 
    \item (b) For $x>0$, $f(x) = \int_{0}^{x} f'(t)dt > 0$. \\
        We can get $\Phi(0) = \Phi(1) = 0$, so $\Phi(x) = 0 \implies \Phi'(x) = 0 \implies 2\int_{0}^{x}f(t)dt = f^2 (x) \\
        \implies f'(x) = 1 \implies f(x) = x + C, for C \in \mathbb{R} $.
    
\end{solution}

\vspace{3cm}

\begin{problem}{2}
    若函数 $f$ 在 $\left[a, b\right]$ 上单调增加, 证明:
    \[
        \int_a^b xf(x) \mathrm{d} x \geq \frac{a+b}{2}\int_a^b f(x) \mathrm{d} x.
    \]
    
\end{problem}


\begin{solution}{2} \textbf{2}

    \begin{itemize}
        \item \textbf{解法一:} 令 $F(x) = \int_a^x tf(t)\mathrm{d}t - \frac{a+b}{2}\int_a^xf(t)\mathrm{d}t.$ \\
            则$F(a) = 0,$ 于是有\\
            \[
                \begin{aligned}
                    F'(x) &= \frac{1}{2} \Big[ (x-a)f(x) - \int_a^x f(x)\mathrm{d}x \Big] \\
                    &= \frac{1}{2}\Big[ f(x)\int_a^x \mathrm{d}t - \int_a^x f(t)\mathrm{d}t \Big] \\
                    &= \frac{1}{2}\int_a^x \Big[f(x) - f(t)\Big]\mathrm{d}t
                \end{aligned}
            \]
            因为 $f$ 在$\left[a, b\right]$上单调增加, $f(x) - f(t)中x\geq t$, 所以$f(x) - f(t)\geq 0$, 所以$F'(x)\geq 0$, 即 $F(x)$单调增加, 所以
            \[
                F(x)\geq F(0) = 0(x\in \left[a, b\right])
            \]
            从而有$F(b)\geq 0$, 即原不等式成立.
        
        \item \textbf{解法二:} 由于$f$在$\left[a, b\right]$上单调增加, 从而对$\forall x, y \in \left[a, b\right]$恒有
            \[
                (f(x) - f(\frac{a+b}{2}))(x - \frac{a+b}{2}) \geq 0   
            \]
            即:
            \[
                xf(x) - \frac{a+b}{2}f(x) - xf(\frac{a+b}{2}) + \frac{a+b}{2}f(\frac{a+b}{2}) \geq 0
            \]
            对两边积分可得:
            \[
                \int_a^b xf(x)\mathrm{d}x - \frac{a+b}{2}\int_a^b f(x)\mathrm{d}x - f(\frac{a+b}{2})\int_a^b x\mathrm{d}x - \frac{a+b}{2}f(\frac{a+b}{2})\int_a^b \mathrm{d}x \geq 0\\
            \]
            \[
                \int_a^b xf(x)\mathrm{d}x - \frac{a+b}{2}\int_a^b f(x)\mathrm{d}x \geq 0    
            \]
            即原不等式成立.
        
        \item \textbf{解法三(积分第二中值定理):} 将不等式的右边移到左边, 然后用积分第二中值定理变形即可.
    \end{itemize}
    
\end{solution}

\vspace{3cm}


\begin{problem}{3}
    设函数$f(x)$为区间$\left[a, b\right]$上的正值连续函数, 且单调递减, 证明:
    \[
        \frac{\int_0^1 xf^2(x)\mathrm{d}x}{\int_0^1 xf(x)\mathrm{d}x} \leq \frac{\int_0^1 f^2(x)\mathrm{d}x}{\int_0^1 f(x)\mathrm{d}x}.
    \]
\end{problem}


\begin{solution}{3} \textbf{3}

    \begin{itemize}
        \item \textbf{解法一:} 由于函数$f(x)$为区间$\left[0, 1\right]$上的正值连续函数, 则
            \[
                \int_0^1 f(x)\mathrm{d}x > 0, \int_0^1 xf(x)\mathrm{d}x > 0    
            \]
            这样只需证明
            \[
                \int_0^1 f^2(x)\mathrm{d}x \int_0^1 xf(x)\mathrm{d}x - \int_0^1 f(x)\mathrm{d}x \int_0^1 xf^2(x)\mathrm{d}x \geq 0,    
            \]
            即证
            \[
                \int_0^1 f^2(x)\mathrm{d}x \int_0^1 yf(y)\mathrm{d}y - \int_0^1 f(x)\mathrm{d}x \int_0^1 yf^2(y)\mathrm{d}y \geq 0,    
            \]
            亦即证
            \[
                \int_0^1 \int_0^1 yf(x)f(y)[f(x) - f(y)]\mathrm{d}x\mathrm{d}y \geq 0.    
            \]
            考虑二重积分
            \[
                I = \int_0^1 \int_0^1 f(x)f(y)(y-x)[f(x) - f(y)]\mathrm{d}x\mathrm{d}y.    
            \]
            因为函数$f(x)$在区间$\left[0, 1\right]$上单调减少, 则对于$\forall x, y\in \left[0, 1\right]$有
            \[
                (y-x)[f(x) - f(y)] \geq 0.    
            \]
            又函数$f(x)$为区间$\left[0, 1\right]$上的正值函数, 则由二重积分的保号性知$I\geq 0$, 又
            \[
                I=\int_0^1 \int_0^1 f(x)f(y)y[f(x)-f(y)]\mathrm{d}x\mathrm{d}y - \int_0^1 \int_0^1 f(x)f(y)x[f(x)-f(y)]\mathrm{d}x\mathrm{d}y,    
            \]
            将上式右边第二项中的x,y对调, 可得
            \[
                \begin{aligned}
                    I&=\int_0^1 \int_0^1 f(x)f(y)y[f(x)-f(y)]\mathrm{d}x\mathrm{d}y - \int_0^1 \int_0^1 f(x)f(y)y[f(y)-f(x)]\mathrm{d}x\mathrm{d}y \\
                    &= 2\int_0^1 \int_0^1 f(x)f(y)y[f(x) - f(y)]\mathrm{d}x\mathrm{d}y.
                \end{aligned}
            \]
            则由$I\geq 0$知原不等式成立.

        \item \textbf{解法二:} 令
            \[
                F(t) = \int_0^1 f^2(x)\mathrm{d}x \int_0^1 xf(x)\mathrm{d}x - \int_0^1 f(x)\mathrm{d}x \int_0^1 xf^2(x)\mathrm{d}x, 0\leq t\leq 1.
            \]
            则
            \[
                \begin{aligned}
                    F'(t) &= f^2(t)\int_0^1 xf(x)\mathrm{d}x + tf(t)\int_0^1 f^2(x)\mathrm{d}x - f(t)\int_0^1 xf^2(x)\mathrm{d}x - tf^2(t)\int_0^1 f(x)\mathrm{d}x \\
                    &= f(t)\int_0^1 (x - t)[f(t) - f(x)]f(x)\mathrm{d}x
                \end{aligned}
            \]
            因$f(x)$为区间$\left[0, 1\right]$上的正值单调减函数, 有$(x-t)[f(t)-f(x)] \geq 0$, 由积分的保号性知$F'(t)\geq 0$, 即$F(t)$在区间$\left[0, 1\right]$上单调增加,
            即$F(1) \geq F(0) = 0$, 即原不等式成立. 
    \end{itemize}
\end{solution}


\vspace{3cm}


\begin{problem}{4}
    
    设$f(x)$在$\left[0, 1\right]$上连续且递减, 证明:
    \[
        \text{当}0<\lambda <1\text{时}, \int_0^{\lambda} f(x)\mathrm{d}x \geq \lambda \int_0^1 f(x)\mathrm{d}x.    
    \]
\end{problem}

\begin{solution}{4} \textbf{4}

    \begin{itemize}
        \item \textbf{解法一:} 令$F(x)=\int_0^x f(t)\mathrm{d}t - x\int_0^1 f(t)\mathrm{d}t, x\in \left(0, 1\right)$, 则$F(0)=0, F(1)=0$.\\
        有
        \[
            \begin{aligned}
                F'(x) &= f(x) - \int_0^1 f(t)\mathrm{d}t \\
                &= f(x) - f(\phi), \phi \in \left(0, 1\right)
            \end{aligned}    
        \]
        于是当$0<x<\phi$时, $F'(x) = f(x)-f(\phi) > 0$, 当$\phi<x<1$时, $F'(x) = f(x)-f(\phi) < 0.$\\
        即$F(x)$在$\left(0, \phi\right)$内单调增加, 在$\left(\phi, 1\right)$内单调减少, 所以$F(x)\geq min{F(0), F(1)} = 0$. 即
        \[
            \int_0^x f(t)\mathrm{d}t - x\int_0^1 f(t)\mathrm{d}t \geq 0, x\in \left(0, 1\right).    
        \]
        得证.
        
        \item \textbf{解法二:} 对$\int_0^{\lambda} f(x)\mathrm{d}x$, 令$x = \lambda t$, 则原不等式可化为
        \[
            \int_0^{\lambda} f(x)\mathrm{d}x = \int_0^1 f(\lambda t)\mathrm{d}\lambda t \geq \lambda \int_0^1 f(x)\mathrm{d}x.
        \]
        即证
        \[
            \lambda \int_0^1 f(\lambda t)\mathrm{d}t \geq \lambda \int_0^1 f(x)\mathrm{d}x    
        \]
        即证$f(\lambda t) \geq f(t)$, 因为$0<\lambda<1$, 所以$\lambda t < t$, 又$f(x)$单调减少, 故结论成立.
    \end{itemize}    
\end{solution}


\begin{problem}{5}
    计算$\int_0^{2\pi} \frac{\sin(2n+1)x}{\sin x}\mathrm{d}x .$
\end{problem}

\begin{solution}{5} \textbf{5}
    
    由欧拉公式$e^{ix} = \cos x + i\sin x$, 则$e^{-ix} = \cos x - i\sin x$. 由以上两个公式, 相减可得
    \begin{equation} \label{}
        sinx = \frac{e^{ix} - e^{-ix}}{2}    
    \end{equation}
    因此, $\sin(2n+1) = \frac{e^{(2n+1)ix} - e^{-(2n+1)ix}}{2}$. 考虑到被积函数为$\frac{\sin(2n+1)}{\sin x}$, 所以
    \[
        \begin{aligned}
            \frac{\sin(2n+1)x}{\sin x} &= \frac{\frac{e^{(2n+1)ix}-e^{-(2n+1)ix}}{2}}{\frac{e^{ix} - e^{-ix}}{2}} \\
            &= \frac{e^{(2n+1)ix - e^{-(2n+1)ix}}}{e^{ix} - e^{-ix}}    
        \end{aligned}
    \]
    利用恒等式
    \[
        a^n - b^n = (a-b)(a^{n-1} + a^{n-2}b + \cdots + ab^{n-2} + b^{n-1})    
    \]
    所以$e^{(2n+1)ix} - e^{-(2n+1)ix}$可以展开为:
    \[
        (e^{ix} - e^{-ix})(e^{2nix} + e^{(2n-1)ix}\cdot e^{-ix} + \cdots + e^{-2nix}).    
    \]
    因此, 原积分可以化为:
    \[
        \begin{aligned}
            &\int_0^{2\pi} \frac{\sin(2n+1)x}{\sin x}\mathrm{d}x \\
            &= \int_0^{2\pi} \left( e^{2nix} + e^{(2n-1)ix}\cdot e^{-ix} + \cdots + e^{-2nix} \right)\mathrm{d}x \\
            &= 2\pi
        \end{aligned}    
    \]
\end{solution}


\section{多元向量代数与空间解析几何}


\section{多元微积分}

\subsection{多元函数微分学}


\subsection{重积分}


\subsection{曲线积分与曲面积分}


\section{无穷级数}

\begin{problem}{1}
设 $(\lambda_n)_{n=1,2,\cdots}$ 是严格单调递增趋于无穷大的正数列。证明:若级数 $\displaystyle{ \sum_{n=1}^{+\infty}}\lambda_n a_n$ 收敛,则 $\displaystyle{\sum_{n=1}^{+\infty}}a_n$ 收敛。
\end{problem}

\begin{solution}{1} \textbf{1.}

\begin{itemize}  

	\item \textbf{解法一(构造法):} 我们令 $\displaystyle{A_n=\sum_{k=1}^{n}\lambda_k a_k}$, $\displaystyle{B_n = \sum_{k=1}^{n}a_k}$, $A_0=0$, 其中$n\geq1$, 则有:
\[
a_k = \frac{A_k - A_{k-1}}{\lambda_k} (k\geq1)
\]

\[
\begin{aligned}
B_n &= \sum_{k=1}^{n}a_k \\
&= \sum_{k=1}^{n} \frac{A_k - A_{k-1}}{\lambda_k}\\
&= \sum_{k=1}^{n-1} (\frac{1}{\lambda_k}-\frac{1}{\lambda_{k+1}}) A_k + \frac{A_n}{\lambda_n}
\end{aligned}
\]
用为 $\displaystyle{\sum_{n=1}^{+\infty}} \lambda_n a_n$ 收敛, 所以$A_n$有界, 又:
\[
\sum_{n=1}^{+\infty}(\frac{1}{\lambda_n} - \frac{1}{\lambda_{n+1}}) = \frac{1}{\lambda_1}
\]
所以级数$\displaystyle{\sum_{n=1}^{\infty}}(\frac{1}{\lambda_n} - \frac{1}{\lambda_{n+1}})A_n$收敛. 由已知条件可知$\lambda_n \to +\infty(n\to\infty)$, 我们可以得到:
\[
\lim_{n\to\infty} \frac{A_n}{\lambda_n} = 0
\]
所以
\[
\begin{aligned}
\sum_{n=1}^{+\infty} a_n 
&= \lim_{n\to\infty} \sum_{k=1}^{n} (\frac{1}{\lambda_k}-\frac{1}{\lambda_{k+1}})A_k+\frac{A_n}{\lambda_n}
&= \sum_{n=1}^{+\infty} (\frac{1}{\lambda_n} - \frac{1}{\lambda{n+1}})A_n
\end{aligned}
\]
故级数$\displaystyle{\sum_{n=1}^{+\infty}}a_n$收敛, 得证.



	\item \textbf{解法二(Abel判别法):} 我们令$A_n = \frac{1}{\lambda_n}(n\geq1)$, 由题意可知数列$A_n$单调递减且有界, 又级数$\displaystyle{\sum_{n=1}^{+\infty}}$收敛, 所以级数
\[
\sum_{n=1}^{+\infty}a_n = \sum_{n=1}^{+\infty}\lambda_n a_n \cdot{} A_n
\]
收敛.

\end{itemize}
\end{solution}

\vspace{3cm}

\begin{problem}{2}
    求级数 $\displaystyle \sum_{n=1}^{+\infty} \frac{(-1)^n}{n \cdot \sqrt[n]{n}} (\frac{x}{2x+1})^n$ 的收敛域.
\end{problem}

\begin{solution}{2} \textbf{2.}

    令 $\displaystyle t = \frac{x}{2x+1}$ ,考察 $\displaystyle \sum_{n=1}^{+ \infty} \frac{(-1)^n}{n \cdot \sqrt[n]{n}} t^n $ 的收敛域. 于是我们由根
    值判别法可得:

    \[
        \begin{aligned}
            \lim_{n \to \infty} \sqrt[n]{\left | \frac{(-1)^n}{n \cdot \sqrt[a]{n}} \right |} &= \lim_{n \to \infty} \frac{1}{n^{\frac{1}{n}}
            \cdot n^{\frac{1}{n^2}}}  \\
            &= \lim_{n \to \infty} \frac{1}{n^{\frac{1}{n} + \frac{1}{n^2}}} \\
            &= \exp(\lim_{n \to \infty} -(\frac{1}{n} + \frac{1}{n^2}) \ln n) \\
            &= 1
        \end{aligned}
    \]

    当 $ x = -1 $ 时, $ \displaystyle \sum_{n=1}^{+ \infty} \frac{(-1)^n}{n \cdot \sqrt[n]{n}} (-1)^n = \sum_{n=1}^{n \to \infty} \frac{1}{n \cdot \sqrt[n]{n}} $,
    而 $ \displaystyle \lim_{n \to \infty} \frac{\frac{1}{n \cdot \sqrt[n]{n}}}  {\frac{1}{n}} = \lim_{n \to \infty} \frac{1}{\sqrt{n}} = 1 $, 从而该级数发散;

    类似可以得到当 $ x = 1 $ 时, $ \displaystyle\sum_{n=1}^{+\infty} \frac{(-1)^n}{n \cdot \sqrt[n]{n}} \cdot (1)^n = \sum_{n=1}^{+\infty}  \frac{(-1)^n}{n \cdot 
    \sqrt[n]{n}} $ 收敛. 
    
    所以级数 $ \displaystyle\sum_{n=1}^{+\infty} \frac{(-1)^n}{n \cdot \sqrt[n]{n}} t^n $ 的收敛域为 $ -1 < x \leq 1 $. 从而有
    $ -1 < \frac{x}{2x+1} \leq 1 $, 解该不等式可得 $ x \leq -1 $ 或 $ x > -\frac{1}{3} $. 所以原级数的收敛域为: $ (-\infty, -1] \cup (-\frac{1}{3}, +\infty)$.

\end{solution}

\vspace{3cm}



\begin{problem}{3}
    设 $f(x)$ 为周期为 $2\pi$ 的连续函数, 令
    \[
        F(x) = \frac{1}{2h} \int_{x-h}^{x+h} f(t) dt.  
    \]
    设 $f(x)$ 的Fourier系数为 $a_0, a_n, b_n(n = 1, 2, \cdots)$, 试求 $F(x)$ 的Fourier系数 $A_0, A_n, B_n(n = 1, 2, \dots)$.


\end{problem}


\begin{solution}{3} \textbf{3}
    
\end{solution}


\vspace{3cm}



\begin{problem}{4}
    求 $ \displaystyle \sum_{n=1}^{10^9} n^{-\frac{2}{3}} $ 的整数部分.
\end{problem}

\begin{solution} \textbf{4}

    记 $I = \sum_{n=1}^{10^9} n^{- \frac{2}{3}}$ , 则由积分与级数之间的关系有:
    \[
        I > \int_{1}^{10^9 + 1} x^{-\frac{2}{3}}dx = 3(\sqrt[3]{1 + 10^9} - 1) > 3(10^3 -1).    
    \]
    \[
        \begin{aligned}
        I - 1 &= \sum_{2}^{10^9} n^{-\frac{2}{3}}dx < \int_{1}^{10^9} x^{-\frac{2}{3}}dx   \\
        &= 3x^{\frac{1}{3}} \Big |_{1}^{10^9} = 3(10^3 -1).
        \end{aligned}
    \]

    即 $I < 3(10^3 -1) + 1$ , 所以 $[I] = 3(10^3 - 1) = 2997$.
\end{solution}

\vspace{3cm}



\pagebreak




\section{微分方程}


\section{Linear Algebra}



\section{真题}





\section{Others}


\begin{problem}{1}
证明Cauthy-Schwarz不等式:
\[
\boldsymbol{\mid<a,b>\mid \leq \parallel a \parallel\parallel b \parallel}
\]

其中 $<\cdot,\cdot>$ 为内积运算.
\end{problem}

\begin{solution}{2}
Cauthy-Schwarz不等式的另一种形式:
\[
\sum_{k=1}^{n} a_kb_k \leq \sum_{k=1}^n a{_k^2} \sum_{k=1}^{n} b{_k^2}
\]
我们令:
\[
A = \sum_{k=1}^{n} a{_k^2}, 
B = \sum_{k=1}^{n} b{_k^2},
C = \sum_{k=1}^{n} a_kb_k
\]
我们只需证明:
\[
C^2 \leq AB
\]
考虑到: $(a+b)^2 \geq 0$ , 有:
\[
0 \leq \sum_{k=0}^n (a_k+tb_k)^2 = A + 2tC + t^2B
\]
当 $B = 0$ 时, 显然 $C = AB$.
当 $B \neq 0$ 时, 由 $\Delta <0$, 可得:
\[
4C^2 - 4AB < 0
\]
\[
C^2 < AB
\]
综上所述:
\[
C^2 \leq AB
\]
得证.
\end{solution}


\vspace{2cm}



\begin{problem}{2}
\begin{enumerate}[label=\alph*)]
    \item Suppose an entire function $f$ is bounded by $M$ along $|z|=R$. Show that the coefficients $C_k$ in its power series expansion about $0$ satisfy
    \[
    |C_k|\leq\frac{M}{R^k}.
    \]
    \item Suppose a polynomial is bounded by $1$ in the unit disc. Show that all its coefficients are bounded by 1.
\end{enumerate}
\end{problem}

%Below is the solution environment
\begin{solution}{}
Part a): Since $f$ is an entire function it can be expressed as an infinite power series, i.e.
\[
f(z)=\sum_{k=0}^\infty\frac{f^{(k)}(0)}{k!}z^k=\sum_{k=0}^\infty C_kz^k.
\]
If we recall Cauchy's Integral we have
\[
f(z)=\frac{1}{2\pi i}\int_\gamma\frac{f(w)}{w-z}\ dw,
\]
carefully notice that $\frac{1}{w-z}=\frac{1}{w}\cdot\frac{1}{1-\frac{z}{w}}$ can be written as a geometric series. We have

%The align environment with no label
\begin{align*}
\frac{1}{2\pi i}\int_\gamma\frac{f(w)}{w-z}\ dw &=\frac{1}{2\pi i}\int_\gamma\left\lbrace\frac{f(w)}{w}\cdot\left(\frac{1}{1-\frac{z}{w}}\right) \right\rbrace\ dw\\[8pt]
&=\frac{1}{2\pi i}\int_\gamma\left\lbrace\frac{f(w)}{w}\cdot\left(1+\frac{z}{w}+\frac{z^2}{w^2}+\frac{z^3}{w^3}+\cdots\right) \right\rbrace\ dw\\[8pt]
&=\left(\frac{1}{2\pi i}\int_\gamma \frac{f(w)}{w}\ dw\right)z^0+\left(\frac{1}{2\pi i}\int_\gamma \frac{f(w)}{w^2}\ dw\right)z^1+\left(\frac{1}{2\pi i}\int_\gamma \frac{f(w)}{w^3}\ dw\right)z^2\cdots
\end{align*}
Now take the modulus of $C_k$ to get
\[
|C_k|=\left\lvert\frac{1}{2\pi i}\int_\gamma \frac{f(w)}{w^{k+1}}\ dw \right\rvert\leq\frac{1}{2\pi}\int_\gamma\frac{|f(w)|}{|w^{k+1}|}\ |dw|\leq \frac{M}{2\pi}\int_\gamma\frac{|dw|}{|w^{k+1}|}
\]
Then integrate along $\gamma(\theta)=Re^{i\theta}$ for $\theta\in [0,2\pi]$ to get
\[
|C_k|\leq \frac{M}{2\pi}\int_0^{2\pi}\frac{|iRe^{i\theta}\ d\theta|}{|R^{k+1}e^{ik\theta}|}=\frac{M}{2\pi\cdot R^k}\int_0^{2\pi}\ d\theta=\frac{M}{R^k}.
\]
Hence, $|C_k|\leq \frac{M}{R^k}$.
\end{solution}
\pagebreak

\end{document}
